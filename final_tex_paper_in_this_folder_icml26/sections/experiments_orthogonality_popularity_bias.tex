\subsection{Does Orthogonality of Embedding Movement Imply Popularity Bias?}

\begin{table*}[!t]
\caption{Last two subsections of experiments — the emergence and detectability of popularity bias. Exp. 8 and 9 report whether displacement is sufficient and quantify popularity bias both as a geometric frequency–norm correlation and as a retrieval skew. The retrieval bias is measured as the share of popular items (from the top-1\% by training frequency) appearing in the top-100 recommendations. Exp. 10 demonstrates that this mechanism persists in a production-like asymmetric setup.\\}\label{tab:orthogonality-popbias}
\centering
\small
\begin{tabular}{@{}ccccccccc@{}}
\toprule
No. & \begin{tabular}{@{}c@{}} Left (user) \\ encoder \end{tabular} & \begin{tabular}{@{}c@{}} Cond.~1 for \\ left encoder? \end{tabular} & Cond.~2? & \begin{tabular}{@{}c@{}} Orthogonal \\ trajectory? \end{tabular} & \begin{tabular}{@{}c@{}} Suff. orthogonal \\ displacement? \end{tabular} & \begin{tabular}{@{}c@{}} Item Pop--Norm \\ Correlation? \end{tabular} & \begin{tabular}{@{}c@{}} Top-1\% \\ (dot) \end{tabular} & \begin{tabular}{@{}c@{}} Top-1\% \\ (cos) \end{tabular} \\
\midrule
8 & Embedding & $\checkmark$ & $\checkmark$ & $\checkmark$ & \begin{tabular}{@{}c@{}} $\checkmark$ \\ (high LR) \end{tabular} & $\checkmark$ & \textbf{0.72} & 0.29 \\
9 & Embedding & $\checkmark$ & $\checkmark$ & $\checkmark$ & \begin{tabular}{@{}c@{}} $\times$ \\ (low LR) \end{tabular} & $\times$ & 0.007 & 0.007 \\
10 & \begin{tabular}{@{}c@{}} BERT-like (with\\Adam optimizer) \end{tabular} & $\times$ & $\checkmark$ & \begin{tabular}{@{}c@{}} User $\times$\\Item $\checkmark$ \end{tabular} & \begin{tabular}{@{}c@{}} $\checkmark$ \\ (high LR) \end{tabular} & $\checkmark$ & \textbf{0.63} & 0.32 \\
\bottomrule
\end{tabular}
\end{table*}

See Table~\ref{tab:orthogonality-popbias}. In this subsection I use MovieLens 32M dataset and the same InfoNCE cosine-based loss.

In Experiments~8 and~9 I test whether the orthogonality property leads to the emergence of popularity bias. 

\textit{Result.} If embeddings move orthogonally under the factors described above, they do increase their norms; thus items that “receive movement” more often (more popular items, appearing more frequently in training batches) systematically accumulate larger norms. I observe a substantial and statistically significant correlation between item popularity and embedding norm (Pearson correlation 0.66; the null hypothesis of zero correlation is rejected at $p<0.05$). However, there is an important caveat, discussed next.

\textit{Nuance.} I identify an important factor required for the appearance of popularity bias under orthogonal movement: embeddings must move with sufficiently large orthogonal displacement. Purely tangential movement over very short distances yields almost no increase in the embedding norm (by the Pythagorean theorem) and cannot overcome the random norm at initialization. In Experiment~8 I increased the learning rate of the SGD optimizer so that objects moved over larger distances; with a commonly used smaller learning rate (LR $=0.1$) objects moved extremely slowly and effectively did not leave their previous-radius hyperspheres, and I did not observe a statistically significant popularity–norm correlation (Experiment~9).

This finding is consistent with the analysis from Section~\ref{sec:norm-monotonicity}.

The last two columns of Table~\ref{tab:orthogonality-popbias} confirm that this geometric phenomenon translates directly into retrieval outcomes. In Experiment~8, where norms grow orthogonally, the share of popular items in the top-100 results is drastically higher for dot-product ranking (72\%) compared to cosine (29\%). Conversely, in Experiment~9, where norms barely change, both ranking methods yield identical distributions ($\approx$0.7\%).

