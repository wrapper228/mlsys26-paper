\subsection{Proposition: When Embeddings Move Orthogonally}

If the encoder satisfies the following four conditions simultaneously:
\begin{enumerate}
  \item optimization uses SGD without momentum and without regularization (the only way to guarantee strict movement defined by equations), 
  \item the encoder is linear in its parameters (parameter-linear),
  \item each distinct input has a dedicated, non-shared parameter row (no overlap across inputs),
  \item the loss gradient with respect to the encoder output (the embedding) is orthogonal to that output,
\end{enumerate}

then it can be proved analytically (\ref{eq:ort-move-proof}) that, during training, the embedding moves orthogonally, i.e., tangentially to the hypersphere on which it located at the previous step.

\begin{table*}[!t]
\centering
\small
\begin{tabular}{@{}cccccc@{}}
\toprule
No. & Left (user) encoder & Loss & \begin{tabular}{@{}c@{}} Cond.~1 for \\ left encoder? \end{tabular} & Cond.~2? & Orthogonal trajectory? \\
\midrule
1 & Embedding & InfoNCE (cos) & $\checkmark$ & $\checkmark$ & $\checkmark$ \\
2 & Embedding & InfoNCE (\textbf{dot}) & $\checkmark$ & $\times$ & $\times$ \\
3 & BERT-like & InfoNCE (cos) & $\times$ & $\checkmark$ & \begin{tabular}{@{}c@{}} User $\times$ / Item $\checkmark$ \end{tabular} \\
4 & Embedding $\to$ Linear & InfoNCE (cos) & $\times$ & $\checkmark$ & \begin{tabular}{@{}c@{}} User $\times$ / Item $\checkmark$ \end{tabular} \\
5 & Embedding (frozen) $\to$ Linear & InfoNCE (cos) & $\times$ & $\checkmark$ & \begin{tabular}{@{}c@{}} User $\times$ / Item $\checkmark$ \end{tabular} \\
6 & one-hot $\to$ Linear & InfoNCE (cos) & $\checkmark$ & $\checkmark$ & $\checkmark$ \\
7 & Embedding & $\bigl(1 - \cos(q,k)\bigr)^{2}$ & $\checkmark$ & $\checkmark$ & $\checkmark$ \\
\bottomrule
\end{tabular}
\caption{Experiments on orthogonality of embedding movement across left (user) encoder architectures and losses. See definitions on ``Cond.~1'' and ``Cond.~2'' in experiments section.}\label{tab:orthogonality-summary}
\end{table*}