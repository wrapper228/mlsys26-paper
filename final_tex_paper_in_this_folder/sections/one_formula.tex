\section{Одна формула, на которой строится весь анализ}
Рассматриваем двухбашенную архитектуру с левым и правым энкодерами. Обозначим для любого примера батча градиент лосса по выходу энкодера как \(g_j = \partial \mathcal{L}/\partial q_j\), а Якобиан выхода по параметрам энкодера как \(J_j = \partial q_j/\partial \theta\). Тогда градиент по параметрам и шаг SGD задают приращение параметров \(\Delta\theta\), а линейная аппроксимация первого порядка даёт приращение выхода для конкретного примера \(i\):
\[
\Delta q_i = J_i\,\Delta\theta = -\eta\sum_j J_i J_j^{\top} g_j.
\]
Эта формула служит отправной точкой дальнейшего анализа условий, когда нормa \(\|q_i\|\) коррелирует с популярностью объекта.
