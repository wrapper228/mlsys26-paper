\subsection{What Is a Parameter-Linear Encoder}

For notation, let
\( q(\theta, x_i) := q_i \) 
denote the encoder output with parameters $\theta$ given input $x_i$ (e.g., categorical feature or feature vector for example $i$), and
\( J(x_i) := J_i \) 
the Jacobian $\partial q(\theta, x_i)/\partial\theta$.

The encoder is \emph{parameter-linear} if its output for any input example $i$ can be written in the exact linear form
\begin{align}
q(\theta, x_i) = J(x_i)\,\theta
\end{align}
where $J(x_i)$ does not depend on $\theta$ and is therefore constant over the parameter space.

Consequently, if the encoder is parameter-linear, then equation \eqref{eq:starting-formula} holds without approximation.