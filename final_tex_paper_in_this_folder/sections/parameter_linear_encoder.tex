\subsection{What Is a Parameter-Linear Encoder}

I call an encoder \emph{parameter-linear} if its output Jacobian with respect to parameters does not depend on the parameters. For clarity of notation, let
\( q_i := q(\theta, x_i) \) denote the encoder output on example $i$ and
\( J_i := J(x_i) \) the Jacobian of the output with respect to $\theta$ evaluated at input $x_i$.

Under this definition the encoder output admits the representation
\begin{align}
q(\theta, x_i) = J(x_i)\,\theta, \label{eq:param-linear-def}
\end{align}
that is, the matrix $J$ may depend on the input (and fixed hyperparameters) but not on $\theta$.

Consequently, $J_i$ is constant over the entire parameter space, and the relation from Section~2.1 becomes \emph{exact}:
\begin{align}
\Delta q_i = J_i\,\Delta\theta. \label{eq:exact-delta-q}
\end{align}
This identity will be used repeatedly in what follows to derive orthogonality and norm-growth properties.


