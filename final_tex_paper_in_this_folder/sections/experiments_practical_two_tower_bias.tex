\subsection{Popularity bias in practical two-tower setup}

Here I consider an asymmetric two-tower model that is close to a production training design: the left tower is a deep BERT (BERT4Rec-like) over the user's history; the right tower is, again, a simple embedding layer over the item identifier. 
Using such a simple architecture for item tower is a popular scenario when representing an item without convolutions and contextual features is an intentional design choice. 
For instance, the YouTube paper adopts such an approach by using a trainable item (class) embedding as the input to a softmax loss; 
the CONTEXTGNN paper likewise employs a simple item-embedding matrix as an item tower, arguing that a complex item-side architecture may not provide substantial gains: \emph{"Limited information gain from applying a GNN on the item side"} and \emph{"Shallow embedding matrices are very effective"}.

I use separate optimizers for the two towers: user BERT is trained by Adam with weight decay, and the item embedding layer is trained by SGD without momentum and without regularization. 
In both cases I use a Cyclic LR schedule. 

Empirical results are consistent with the theory: when the right (item) tower and its training dynamics remain simple -- note that a dynamic learning-rate schedule does not invalidate the orthogonal‑movement result -- item embeddings continue to move orthogonally even in the presence of a complex left‑tower architecture.
Given sufficiently large orthogonal displacement (e.g., LR=10 for item tower), a statistically significant popularity bias emerges, observed as a positive correlation between an item’s embedding norm and its frequency in the training set (Pearson correlation 0.56; the null hypothesis of zero correlation is rejected at $p<0.05$).


