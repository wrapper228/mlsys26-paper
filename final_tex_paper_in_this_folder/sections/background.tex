\section{Предыстория}
В ходе разработки DSSM/двухбашенных моделей для рекомендаций обнаружен устойчивый рост норм эмбеддингов популярных объектов при обучении с косинусными/контрастивными целевыми функциями (InfoNCE, full softmax). Эффект усиливается с частотой встречаемости и влияет на качество ранжирования, особенно при использовании скалярного произведения.

\section{Почему это критично для индустрии?}
\begin{itemize}
  \item Ранжирование часто строится по скалярному произведению \(\langle q_u, q_i\rangle\), что делает итоговый скор зависящим от нормы \(\|q_i\|\).
  \item Систематическое искажение: рост \(\|q_i\|\) у популярных объектов повышает их скор, усиливая их экспозицию.
  \item Эффект ``богатые богатеют'': больше показов \(\Rightarrow\) ещё больший рост норм.
  \item Подавление long-tail: нишевые объекты оказываются недопредставленными.
\end{itemize}

\section{Что вы найдёте в статье}
Мы объясняем:
\begin{itemize}
  \item Как архитектурные свойства энкодеров влияют на динамику эмбеддингов;
  \item Аналитическую зависимость нормы эмбеддинга от популярности объекта;
  \item Условия возникновения popularity bias в практических двухбашенных моделях;
  \item Почему некоторые работы (например, NormFace) упускают важные факторы;
  \item Реплицируемые эксперименты, подтверждающие выводы.
\end{itemize}

\section{О чём статья?}
Мы формализуем причины, по которым при обучении двухбашенных моделей с косинусными целевыми функциями наблюдается рост норм эмбеддингов популярных объектов, и выводим условия возникновения этого явления из анализа градиентной динамики.
