% MLSys 2025 paper skeleton (reset)
\documentclass{article}

% Recommended packages
\usepackage{microtype}
\usepackage{graphicx}
\usepackage{subfigure}
\usepackage{hyperref}
\usepackage{amsmath,amssymb}
\usepackage{booktabs}

% MLSys style
\usepackage{mlsys2025}
% \usepackage[accepted]{mlsys2025}

% Running title (short)
\mlsystitlerunning{Theoretical Grounds for Popularity Bias in Two-Tower Models}

\begin{document}

\twocolumn[
\mlsystitle{Theoretical Grounds for Popularity Bias in Two-Tower Models Trained with Cosine-Based Loss}

% Author and affiliation block (kept in source; hidden in blind review)
\mlsyssetsymbol{equal}{*}
\begin{mlsysauthorlist}
\mlsysauthor{Andrey Atamanyuk}{wb}
\end{mlsysauthorlist}

\mlsysaffiliation{wb}{Wildberries and Russ, Russia}
\mlsyscorrespondingauthor{Andrey Atamanyuk}{atamanyuk.andrey@rwb.ru}

% Keywords are optional
\mlsyskeywords{Representation learning, Two-tower models, Embedding norms}

\vskip 0.3in
\begin{abstract}
    Two-tower models may exhibit popularity bias, observed as a positive correlation between item frequency and its embedding norm that inflates dot-product scores at ranking time. This phenomenon is due to specific properties of loss and encoder architecture. Using analysis of encoder parameters’ updates, I identify sufficient conditions under which embedding updates are provably orthogonal to the current embedding and hence monotonically increase its norm. My theory yields explicit guarantees for the emergence of popularity bias in practical two-tower setups (InfoNCE/cosine-based loss, asymmetric two-tower architecture) and corrects a common misconception in prior works that orthogonality of the gradient alone implies norm inflation for deep encoders.  Empirical studies support the theory via geometry-first investigation: configurations that satisfy these premises exhibit strictly orthogonal embedding movements and a robust statistically significant frequency–norm coupling, whereas violations of any premise break orthogonality and yield non-systematic, noisy update trajectories. The results provide theoretical grounds for popularity bias in cosine-trained two-tower models and show when it should be expected in production systems.
\end{abstract}
]

% Footnote line (hidden unless [accepted])
\printAffiliationsAndNotice{\mlsysEqualContribution}

% Introduction
\section{Introduction}
Two-tower architectures are the de facto standard for large-scale retrieval and recommendation \cite{huang2024retrievalsurvey}. Their efficiency stems from decoupling the user and item encoders and scoring by a simple similarity, which enables precomputation, approximate nearest-neighbor search, and low-latency serving at web-scale. Yet the role of embedding magnitudes in training and ranking remains under-characterized: empirical observations vary across systems, and theory has not reached consensus on when and why norms change under different objectives and optimizers.

From the definition of cosine similarity, \(\cos(q,k)=\langle q,k\rangle/(\|q\|\,\|k\|)\), the normalization appears to cancel the effect of magnitudes, so objectives that couple the left and right encoder embeddings only through cosine similarity are expected to be insensitive to embedding norms. A competing line of analysis argues \cite{wang2017normface,draganov2024pitfalls,draganov2025embeddingnorms} that for cosine-based losses the gradient \(\nabla_{q}\,\mathcal{L}\) is orthogonal to \(q\), and therefore any gradient step must increase \(\|q\|\), often extrapolated to deep encoders. Both perspectives ignore how parameter updates propagate through the encoder and do not explicitly state the algorithmic and architectural conditions under which norm growth necessarily occurs.

We replace black-box reasoning with an update-level analysis. We track how parameter updates map to item-embedding displacements and prove that, under explicit and testable premises — the item encoder is linear in its parameters with non-shared per-item parameters; the item side is trained by plain SGD; and the loss uses the item embedding only via cosine similarities — each step moves the embedding orthogonally to its current value and thus monotonically increases its squared norm. Building on this, we formalize a frequency--norm \emph{mechanism (conditional on per-update increments)}: popularity increases how often item updates occur, and orthogonality ensures each such update contributes a nonnegative radial increment to $\|q_i\|^2$ (Appendix~\ref{app:popularity-dependence}).

The analysis also delineates the limits of claims that ``gradient orthogonality implies norm inflation'' for deep encoders \cite{wang2017normface,draganov2024pitfalls,draganov2025embeddingnorms}. Orthogonality of the output gradient is insufficient on its own; orthogonality of the \emph{update} follows only when the encoder's Jacobian preserves it, which holds precisely under the stated premises. Violating any premise — adding nonlinearities or parameter sharing in the item tower or optimizing dot product — breaks the guarantee and leads to non-systematic norm dynamics. Our contributions are:

\begin{itemize}
    \item \textbf{Explanation of sufficient conditions for orthogonality of embedding updates.} We prove that 
    embeddings move orthogonally under the following premises: (A1) the item side is optimized by 
    SGD without regularization and momentum; (A2) the item encoder is linear in its parameters; 
    (A3) per-item parameters are not shared; and (A4) the loss depends on the item only via cosine. 
    Orthogonality breaks if any premise is violated.
    \item \textbf{Explanation of why orthogonality of embedding updates implies popularity bias.} We prove a formal mechanism linking sampling frequency to norm growth: under orthogonal updates, each update increases $\|q_i\|^2$, and a clean coupling model yields monotonicity of the expected squared norm with sampling probability. Crucially, 
    we demonstrate that this geometric artifact translates directly into popularity biased retrieval 
    outcomes.
    \item \textbf{Clarification of prior claims in the literature.} We re-examine claims from studies on similar 
    topic \cite{wang2017normface,draganov2024pitfalls,draganov2025embeddingnorms} and argue that gradient 
    update affects the model parameters, not its output itself.
    \item \textbf{The ``Baseline Paradox'' in architectural design.} We identify a structural paradox where 
    simple baselines, favored for their transparency in early development, are mathematically guaranteed to 
    exhibit bias. This creates a risk of false negatives (discarding viable projects due to optimization 
    artifacts), though complex production setups remain vulnerable too.
\end{itemize}

\paragraph{Scope and implications.} We establish sufficient conditions under which embedding updates are orthogonal and popularity bias emerges. Outside this regime, these effects may or may not occur; we make no claim of necessity. Other sources of popularity bias are examined in prior work \cite{zhang2023dimensional}.
\section{Sufficient Conditions for Orthogonality of Embedding Updates}

\section{One formula that underpins the analysis}
---


\subsection{Interim Focus: Parameter-Linear Encoders}

The equation developed above is a \emph{linear approximation} of the embedding dynamics during training. For parameter-linear encoders this linearization is exact: for each example $i$ the map from parameters to the output satisfies $q_i(\theta)=J_i\,\theta$, and hence equation \eqref{eq:starting-formula} holds without approximation.

For non-linear encoders, the relation $\Delta q_i \approx J_i\,\Delta\theta$ remains accurate under sufficiently small learning rates. I postpone a full treatment of the additional constraints imposed by non-linear architecture to Section~\ref{sec:beyond-linear} (\emph{Beyond linear encoders}).



\subsection{What Is a Parameter-Linear Encoder}

I call an encoder \emph{parameter-linear} if its output Jacobian with respect to parameters does not depend on the parameters. For clarity of notation, let
\( q_i := q(\theta, x_i) \) denote the encoder output on example $i$ and
\( J_i := J(x_i) \) the Jacobian of the output with respect to $\theta$ evaluated at input $x_i$.

Under this definition the encoder output admits the representation
\begin{align}
q(\theta, x_i) = J(x_i)\,\theta, \label{eq:param-linear-def}
\end{align}
that is, the matrix $J$ may depend on the input (and fixed hyperparameters) but not on $\theta$.

Consequently, $J_i$ is constant over the entire parameter space, and the relation from Section~2.1 becomes \emph{exact}:
\begin{align}
\Delta q_i = J_i\,\Delta\theta. \label{eq:exact-delta-q}
\end{align}
This identity will be used repeatedly in what follows to derive orthogonality and norm-growth properties.





\section{Embedding-Norm Is Monotonic in Item Popularity}

Throughout this section (as well as corresponding appendices) I assume the four conditions from Section~2 hold (SGD without momentum/regularization; parameter-linear encoder; non-shared parameter rows; cosine-loss gradient orthogonal to the output). Under these assumptions the embedding moves orthogonally at each update. By the Pythagorean theorem,
\begin{equation}
\label{eq:pythagoras-increment}
\bigl\|q_i^{(t+1)}\bigr\|^{2} - \bigl\|q_i^{(t)}\bigr\|^{2} \;=\; \bigl\|\Delta q_i^{(t)}\bigr\|^{2}.
\end{equation}

Before analyzing the dependence on popularity, I state a claim and provide its proof in the appendix:
\begin{quote}
\textbf{Claim.} The larger the embedding norm, the slower it grows under a cosine-based loss.
\end{quote}
The proof (Appendix~\ref{app:cosine-growth-rate}) derives an explicit expression for the norm of the gradient and shows its inverse dependence on the current norm of $q$.
I use this claim in the following Appendix~\ref{app:popularity-dependence}.

Finally, I analyze the dependence of the embedding-norm growth on item popularity. Using a coupling probabilistic technique (Appendix~\ref{app:popularity-dependence}), I show that the expected embedding norm after $T$ batches is nondecreasing in the item's sampling probability.

\paragraph{Takeaway.} This section and the corresponding appendices provide a formal mechanism for popularity bias: under the four assumptions of Section~2, more frequently sampled items accumulate larger norms. 
To observe a statistically significant popularity bias in practice, an additional (fifth) factor is required; it will be discussed in the next section.



\section{Experiments: Empirical Validation of Theoretical Results}
\label{sec:experiments}

\begin{table*}[!t]
\caption{First subsection of experiments — orthogonality of embedding movement across left (user) encoder architectures and losses. See definitions on ``Cond.~1'' and ``Cond.~2'' in subsection definitions.\\}\label{tab:orthogonality-summary}
\centering
\small
\begin{tabular}{@{}cccccc@{}}
\toprule
No. & Left (user) encoder & Loss & \begin{tabular}{@{}c@{}} Cond.~1 for \\ left encoder? \end{tabular} & Cond.~2? & Orthogonal trajectory? \\
\midrule
1 & Embedding & InfoNCE (cos) & $\checkmark$ & $\checkmark$ & $\checkmark$ \\
2 & Embedding & InfoNCE (\textbf{dot}) & $\checkmark$ & $\times$ & $\times$ \\
3 & BERT-like & InfoNCE (cos) & $\times$ & $\checkmark$ & \begin{tabular}{@{}c@{}} User $\times$ / Item $\checkmark$ \end{tabular} \\
4 & Embedding $\to$ Linear & InfoNCE (cos) & $\times$ & $\checkmark$ & \begin{tabular}{@{}c@{}} User $\times$ / Item $\checkmark$ \end{tabular} \\
5 & Embedding (frozen) $\to$ Linear & InfoNCE (cos) & $\times$ & $\checkmark$ & \begin{tabular}{@{}c@{}} User $\times$ / Item $\checkmark$ \end{tabular} \\
6 & one-hot $\to$ Linear & InfoNCE (cos) & $\checkmark$ & $\checkmark$ & $\checkmark$ \\
7 & Embedding & $\bigl(1 - \cos(q,k)\bigr)^{2}$ & $\checkmark$ & $\checkmark$ & $\checkmark$ \\
\bottomrule
\end{tabular}
\end{table*}

To clarify scope, I do not assess recommendation quality (e.g., MAP@k), given that popularity bias can be either harmful or beneficial depending on the application context \citep{klimashevskaia2024popularitybias}. Such evaluation lies outside the contribution of this work. Instead, I empirically examine the full causal chain predicted by the theory: from the orthogonality of updates and frequency–norm coupling (geometric properties) to the resulting skewed retrieval outcomes.
All datasets and implementation details are available in a public \href{https://docs.google.com/anonymous_link_placeholder}{GitHub repository}.

\subsection{Under Which Factors Orthogonality of Embedding Movement Emerges}

See Table~\ref{tab:orthogonality-summary}. In this subsection we use a small toy synthetic dataset in which personalized dependencies can be observed visually.

\textit{Definitions:}\\
Condition~1: the encoder architecture consists of either (i) a single linear layer applied to one-hot inputs, or (ii) a single embedding layer.\\ 
Condition~2: the gradient of the loss with respect to the encoder output is orthogonal to that output.

\textit{Clarifications.} In all experiments the right tower (item tower) is a simple embedding layer (nn.Embedding); InfoNCE \cite{oord2018cpc} denotes the contrastive softmax loss with in-batch negatives, where ``cos'' (resp., ``dot'') refers to cosine (resp., dot-product) similarity. We use SGD without momentum and without regularization; alternative optimizers did not reveal any movement patterns, although we will separately consider mixed-optimizer settings, including Adam, later (see Experiment~10). In Experiment~4 the architecture coincides with one-hot $\to$ Linear $\to$ Linear, i.e., simply two linear layers. The architecture in Experiment~5 is an example of an encoder that consists of a single Linear layer without the input-orthogonality requirement (no one-hot inputs): the inputs are just some numeric features.

\paragraph{If the encoder is nonlinear in the parameters:}
Recall that two linear layers constitute nonlinearity with respect to the model parameters. We argued that due to this nonlinearity one cannot obtain an analytical explanation of how exactly embeddings move. The result of Experiment~4 is consistent with this reasoning: unlike Experiments~1 and~6, here the embeddings from the left encoder move chaotically (see Figure~\ref{fig:exp4-angles}).

\begin{figure}[!htbp]
\centering
\includegraphics[width=.95\columnwidth]{figures/figure_3_paper.pdf}
\caption{Experiment~4: User-update angles distribution on semicircle; black arrow is user embedding, green arrow is orthogonal direction.}
\label{fig:exp4-angles}
\end{figure}

\paragraph{If, in a parameter-linear encoder, the inputs are not mutually orthogonal:}
Recall that linearity of the encoder is still insufficient to guarantee orthogonal movement of embeddings, and consider the result of Experiment~5: embeddings produced by the left encoder move chaotically (see Figure~\ref{fig:exp5-angles}). This agrees with our statement that, for orthogonality of embedding movement, each unique input to the encoder must have its own parameter row that does not intersect with others. We were able to identify only two (essentially identical) architectures where this holds — a linear layer over one-hot vectors (Experiment~6) or an embedding layer (Experiment~1).

\begin{figure}[!htbp]
\centering
\includegraphics[width=.95\columnwidth]{figures/figure_4_paper.pdf}
\caption{Experiment~5: User-update angles distribution on semicircle; black arrow is user embedding, green arrow is orthogonal direction.}
\label{fig:exp5-angles}
\end{figure}

\paragraph{If a different loss is used:}
The result of Experiment~7 does not contradict our statement that the gradient of any cosine-based loss with respect to the encoder output is orthogonal to that output.

Overall, these experiments confirm that strict orthogonality requires precise conditions: violations lead to non-systematic, chaotic movement. For example, the angle distribution for Experiments~4 and~5 (Figures~\ref{fig:exp4-angles} and~\ref{fig:exp5-angles}) shows no strong concentration around $90°$ (share within $\pm 10°$: $\approx$20\%) and spans the full range $[0°, 180°]$.
\subsection{Does Orthogonality of Embedding Movement Imply Popularity Bias?}

See Table~\ref{tab:orthogonality-popbias}. In this subsection I use MovieLens 32M dataset.

In Experiments~8 and~9 I test whether the orthogonality property leads to the emergence of popularity bias. 

\textit{Result.} If embeddings move orthogonally under the factors described above, they do increase their norms; thus items that “receive movement” more often (more popular items, appearing more frequently in training batches) systematically accumulate larger norms. We observe a substantial and statistically significant correlation between item popularity and embedding norm (Pearson correlation 0.54; the null hypothesis of zero correlation is rejected at $p<0.05$). However, there is an important caveat, discussed next.

\textit{Nuance.} We identify an important factor required for the appearance of popularity bias under orthogonal movement: embeddings must move with sufficiently large orthogonal displacement. Purely tangential movement over very short distances yields almost no increase in the embedding norm (by the Pythagorean theorem) and cannot overcome the random norm at initialization. In Experiment~8 I increased the learning rate of the SGD optimizer so that objects moved over larger distances; with a commonly used smaller learning rate (LR $=0.1$) objects moved extremely slowly and effectively did not leave their previous-radius hyperspheres, and I did not observe a statistically significant popularity–norm correlation (Experiment~9).

This finding is consistent with the analysis from Section~\ref{sec:norm-monotonicity}.


\subsection{Popularity bias in practical two-tower setup}

Here I consider an asymmetric two-tower model that is close to a production training design (Experiment 10, Table~\ref{tab:orthogonality-popbias}, same MovieLens 32M dataset): the left tower is a deep BERT over the user's history; the right tower is, again, a simple embedding layer over the item identifier. 
Using such a simple architecture for item tower is a popular scenario when representing an item without convolutions and content features is an intentional design choice. 
For instance, the YouTube paper adopts such an approach by using a trainable item (class) embedding as the input to a softmax loss; 
the CONTEXTGNN paper likewise employs a simple item-embedding matrix as an item tower, arguing that a complex item-side architecture may not provide substantial gains: \emph{"Limited information gain from applying a GNN on the item side"} and \emph{"Shallow embedding matrices are very effective"}.

I use separate optimizers for the two towers: user BERT is trained by Adam with weight decay, and the item embedding layer is trained by SGD without momentum and without regularization. 
In both cases I use a Cyclic LR schedule. 

Empirical results are consistent with the theory: when the right (item) tower and its training dynamics remain simple -- note that a dynamic learning-rate schedule does not invalidate the orthogonal‑movement result -- item embeddings continue to move orthogonally even in the presence of a complex left‑tower architecture.
Given sufficiently large orthogonal displacement (e.g., LR=10 for item tower), a statistically significant popularity bias emerges, observed as a positive correlation between an item’s embedding norm and its frequency in the training set (Pearson correlation 0.56; the null hypothesis of zero correlation is rejected at $p<0.05$).





% References
\bibliography{references}
\bibliographystyle{mlsys2025}

\appendix
\section{Encoder Examples: Parameter-Linear and Nonlinear}\label{app:encoders}

\subsection{Encoder with a Single Embedding Layer}\label{app:embed}
\textbf{Input:} $x_i = i \in \{1,\dots,N\}$ \;($x_i$ -- index of example $i$).

\textbf{Parameters:} $E\in\mathbb{R}^{N\times d}$, \;$\theta=E$, \;$\theta_{i,n}=E_{i,n}$ \;(here $n$ is the column index).

\textbf{Output:} $q(\theta,x_i)=E_i\in\mathbb{R}^{d}$, \;$q_i = q(\theta,x_i)$.

Let $m$ denote an output coordinate of $q$, and let $i'$ and $n'$ be arbitrary row/column indices with respect to which the derivative is taken (they range over all parameters). I also vectorize $\theta$ from an $N\times d$ matrix into a vector of length $N\!\cdot\! d$.

\textbf{Jacobian:}
\begin{align}
J_i \,=\, \frac{\partial q_i}{\partial \theta} \,=\, \begin{cases}
1, & \text{if } i' = i,\; n' = m,\\
0, & \text{otherwise.}
\end{cases}
\end{align}

Here $J_i$ is a $d \times (N\!\cdot\! d)$ table that shows how each output coordinate changes under an infinitesimal change of each parameter at fixed input $x_i$.

Equivalently, all entries are zero except the $d\times d$ block corresponding to row $i$, which is the identity $I_d$:
\begin{align}
J_i \,=\, \Bigl[\, \underbrace{0_{d\times (i-1)d}}_{\text{params }1\ldots(i-1)} \;\Big|\; \underbrace{I_d}_{\text{params of row }i} \;\Big|\; \underbrace{0_{d\times (N-i)d}}_{\text{params }(i+1)\ldots N} \Bigr].
\end{align}
Thus $J$ does not depend on $\theta$, and the encoder is parameter-linear.

\subsection{Encoder with a Single Linear Layer (No Bias)}\label{app:linear}
\textbf{Input:} $x_i = (x_{i,1},\dots,x_{i,h})^{\!\top} \in \mathbb{R}^{h}$ \;($x_i$ -- feature vector of example $i$).

\textbf{Parameters:} $W\in\mathbb{R}^{d\times h}$, \;$\theta=W$, \;$\theta_{m,n}=W_{m,n}$ \;(here $m$ indexes output rows and $n$ indexes input features).

\textbf{Output:} $q(\theta,x_i)=W\,x_i\in\mathbb{R}^{d}$, \;$q_i=q(\theta,x_i)$.

Let $m$ denote an output coordinate of $q_i$, and let $m'$ and $n'$ be arbitrary row/column indices with respect to which the derivative is taken (they range over all parameters). I also vectorize $\theta$ from a $d\!\times\!h$ matrix into a vector of length $d\!\cdot\!h$.

\textbf{Jacobian:}
\begin{align}
J_i \,=\, \frac{\partial q_i}{\partial \theta} \,=\, \begin{cases}
x_{i,n'}, & \text{if } m' = m,\\
0, & \text{otherwise.}
\end{cases}
\end{align}

Here $J_i$ is a $d \times (d\!\cdot\! h)$ table showing how each output coordinate changes under an infinitesimal change of each parameter at fixed input $x_i$. Equivalently, in each of the $d$ blocks (width $h$) on row $m$ there is a copy of $x_i^{\!\top}$, the remaining entries are zero:
\begin{align}
J_i \,=\, \operatorname{diag}\!\bigl(\underbrace{x_i^{\!\top},\, x_i^{\!\top},\, \dots,\, x_i^{\!\top}}_{d\ \text{times}}\bigr) .
\end{align}
Hence $J$ depends on the input $x_i$ but not on $\theta$, so this encoder is parameter-linear. If a bias is present, append an identity block on the right.

\subsection{Encoder with Two Consecutive Linear Layers (No Bias)}\label{app:two-linear}
\textbf{Input:} $x_i = (x_{i,1},\dots,x_{i,h})^{\!\top} \in \mathbb{R}^{h}$ \;($x_i$ -- feature vector of example $i$).

\textbf{Parameters:} $A\in\mathbb{R}^{d_1\times h}$, $B\in\mathbb{R}^{d_2\times d_1}$, \;$\theta=\{A,B\}$, \;$\theta^{(A)}_{r,s}=A_{r,s}$, \;$\theta^{(B)}_{m,n}=B_{m,n}$ \;(here $r$ is a “hidden” coordinate, $m$ an output coordinate).

\textbf{Output:} $z_i = A\,x_i \in \mathbb{R}^{d_1}$, \; $q_i = q(\theta,x_i) = B\,z_i \in \mathbb{R}^{d_2}$.

Let $m$ denote an output coordinate of $q_i$. Let $m',n'$ be arbitrary row/column indices in $B$, and $r',s'$ indices in $A$.

\textbf{W.r.t. $B$:}
\begin{align}
\frac{\partial q_{i,m}}{\partial B_{m',n'}} 
= \begin{cases} z_{i,n'}, & m' = m, \\ 0, & \text{otherwise.} \end{cases}
\end{align}

\textbf{W.r.t. $A$:}
\begin{align}
\frac{\partial q_{i,m}}{\partial A_{r',s'}} = B_{m,r'}\,x_{i,s'} .
\end{align}

Therefore,
\begin{align}
J_i = \bigl[\, J^{(B)}_i \;\big|\; J^{(A)}_i \bigr],
\end{align}
where $J^{(B)}_i\in\mathbb{R}^{d_2\times(d_2 d_1)}$ has, in each of the $d_2$ blocks of width $d_1$, the row $z_i^{\!\top}$, and $J^{(A)}_i\in\mathbb{R}^{d_2\times(d_1 h)}$ consists of $d_1 h$ columns of the form $B_{:,r'}\,x_{i,s'}$.

Consequently, $J_i$ depends on the parameters (through $B$), so this encoder is \emph{not} parameter-linear. Intuitively, although the encoder is linear in the \emph{input}, changing a parameter in the first linear layer affects the output through a multiplication by the second layer, which makes the Jacobian parameter-dependent; hence strict parameter-linearity fails.


\section{Cosine Orthogonality Lemma}
\label{app:cos-orth-lemma}

For any nonzero $q,k\in\mathbb{R}^d$,
\begin{align}
q^{\!\top}\,\nabla_q\,\cos(q,k) \,=\, 0.
\end{align}

\noindent\textit{Proof.} Let
\begin{equation}
\begin{aligned}
\cos(q,k) = f(q) = \frac{u(q)}{v(q)}, \\
u(q)=q^{\!\top}k,\; v(q)=\|q\|\,\|k\|.
\end{aligned}
\end{equation}
By the quotient rule,
\begin{align}
\nabla_q f \,=\, \frac{v\,\nabla_q u - u\,\nabla_q v}{v^2}.
\end{align}
The terms are
\begin{align}
\nabla_q u \,=\, k, \qquad \nabla_q v \,=\, \frac{\|k\|}{\|q\|}\, q.
\end{align}
Substituting and simplifying,
\begin{equation}
\begin{aligned}
\nabla_q f 
&= \frac{\|q\|\,\|k\|\,k - (q^{\!\top}k)\,\frac{\|k\|}{\|q\|}\, q}{\bigl(\|q\|\,\|k\|\bigr)^2} \\
&= \frac{1}{\|q\|\,\|k\|}\Bigl(k - \frac{q^{\!\top}k}{\|q\|^2}\, q\Bigr).
\end{aligned}
\end{equation}
Thus
\begin{equation}
\begin{aligned}
q^{\!\top}\nabla_q f 
&= \frac{1}{\|q\|\,\|k\|}\Bigl(q^{\!\top}k - \frac{q^{\!\top}k}{\|q\|^2}\, q^{\!\top}q\Bigr) \\
&= \frac{1}{\|q\|\,\|k\|}\Bigl(q^{\!\top}k - \frac{q^{\!\top}k}{\|q\|^2}\, \|q\|^2\Bigr) \\
&= 0. \qquad
\end{aligned}
\end{equation}



\section{Gradient Magnitude under Cosine-Based Loss}
\label{app:cosine-growth-rate}

\textbf{The bigger $q$, the smaller the norm of the gradient of a cosine-based loss with respect to $q$.}

\emph{Why this matters.} This creates a counter-effect: “the larger the embedding, the slower it grows.” This effect will be used in Appendix~\ref{app:popularity-dependence} for analyzing norm growth of popular items.

\medskip
\noindent\textbf{Derivation.} I begin with the formula from Appendix~\ref{app:cos-orth-lemma}:
\begin{equation}
\label{eq:grad-cos-base}
\begin{aligned}
\nabla_{q}\,\cos(q,k)
&= \frac{\|q\|\,\|k\|\,k - (q^{\!\top}k)\,\|k\|\,\tfrac{q}{\|q\|}}{\bigl(\|q\|\,\|k\|\bigr)^2} \\
&= \frac{k - \tfrac{q^{\!\top}k}{\|q\|^2}\,q}{\|q\|\,\|k\|}.
\end{aligned}
\end{equation}
Introduce
\begin{equation}
\hat q \,=\, \frac{q}{\|q\|},\qquad \hat k \,=\, \frac{k}{\|k\|}.
\end{equation}
Then
\begin{equation}
\begin{aligned}
&\nabla_{q}\,\cos(q,k)
= \frac{1}{\|q\|}\,\bigl(\hat k - (\hat q^{\!\top}\hat k)\,\hat q\bigr) \\
&= \frac{1}{\|q\|}\,(I - \hat q\,\hat q^{\!\top})\,\hat k,\\ &\text{denote } P = I - \hat q\,\hat q^{\!\top}.
\end{aligned}
\end{equation}

Consider $P$. Being a matrix, it is a linear mapping. Since it is idempotent ($P^2=P$) and symmetric ($P^{\!\top}=P$), it is not only linear but, by definition, an orthogonal projector. This projector maps any vector (on which it acts by left multiplication) to the subspace consisting of all vectors orthogonal to $q$.

Now consider an arbitrary cosine-based loss
\begin{equation}
L(q) \,=\, F\big(\cos(q,k_1),\dots,\cos(q,k_m)\big).
\end{equation}

\noindent\textbf{1. Notation.}
\begin{equation}
\begin{aligned}
c_i(q) &= \cos(q,k_i) = \frac{q^{\!\top}k_i}{\|q\|\,\|k_i\|}, \\
\hat q &= \frac{q}{\|q\|}, \qquad \hat k_i = \frac{k_i}{\|k_i\|}, \\
P &= I - \hat q\,\hat q^{\!\top}.
\end{aligned}
\end{equation}

\noindent\textbf{2. Gradient of each cosine.}
\begin{equation}
\nabla_q c_i(q) \,=\, \frac{1}{\|q\|}\,P\,\hat k_i.
\end{equation}

\noindent\textbf{3. Chain rule.}
\begin{equation}
\nabla_q L(q) \,=\, \sum_{i=1}^{m} \frac{\partial F}{\partial c_i}\,\nabla_q c_i(q) \,=\, \frac{1}{\|q\|}\,\sum_{i=1}^{m} \frac{\partial F}{\partial c_i}\,P\,\hat k_i.
\end{equation}

\noindent\textbf{4. Simplification.}
\begin{equation}
\begin{aligned}
\nabla_q L(q)
&= \frac{1}{\|q\|}\,P\left( \sum_{i=1}^{m} \frac{\partial F}{\partial c_i}\,\hat k_i \right), \\
u &= \sum_{i=1}^{m} \frac{\partial F}{\partial c_i}\,\hat k_i, \\
\Rightarrow\; \nabla_q L(q) &= \frac{1}{\|q\|}\,P u.
\end{aligned}
\end{equation}

\noindent\textbf{Norm of the gradient.}
\begin{equation}
\bigl\|\nabla_q L(q)\bigr\| \,=\, \frac{1}{\|q\|}\,\bigl\|P u\bigr\|.
\end{equation}
Because $P$ is an orthogonal projector ($P^{\!\top}=P$, $P^2=P$),
\begin{equation}
\bigl\|P u\bigr\|^{2} \,=\, u^{\!\top} P u \,=\, u^{\!\top}(I - \hat q\,\hat q^{\!\top})u \,=\, \|u\|^{2} - (\hat q^{\!\top}u)^{2}.
\end{equation}
Hence,
\begin{equation}
\begin{aligned}
&\boxed{\; \bigl\|\nabla_q L(q)\bigr\| \,=\, \frac{\sqrt{\|u\|^{2} - (\hat q^{\!\top}u)^{2}}}{\|q\|} \;} , \\
&\text{where }\; u = \sum_{i=1}^{m} \frac{\partial F}{\partial c_i}\,\hat k_i, \quad c_i = \hat q^{\!\top}\hat k_i.
\end{aligned}
\end{equation}

Here I note that, as the embedding norm grows, the gradient norm decreases:
\begin{itemize}
  \item the denominator is the embedding norm $\|q\|$;
  \item the numerator does not depend on $\|q\|$: when computing $u$, $q$ appears only inside the derivative $\partial F/\partial c_i$ (through the cosines). A pure rescaling of $q$ does not change the cosines; therefore the input to $F$ does not change, and the derivatives $\partial F/\partial c_i$ do not depend on $\|q\|$.
\end{itemize}

\section{Popularity Dependence via Coupling}
\label{app:popularity-dependence}

Consider two training processes that are identical except that in one process the item $i$ is sampled with probability $p_i$ and in the other with a higher probability $p_i' > p_i$. Couple the draws so that every occurrence in the lower-probability process is matched in the higher-probability process. Under orthogonal updates (Section~2), each matched occurrence yields an identical norm increment by Eq.~\eqref{eq:pythagoras-increment}, and the higher-probability process accrues weakly more such increments in expectation. Therefore the expected norm after $T$ steps is nondecreasing in $p_i$.



\section{Note on distributions}
\label{app:note-on-distributions}

Let the base distribution for all items be $p$. For the set of non-$i$ items ($k\ne i$) define the residual distribution $\boldsymbol r$ obtained by removing item $i$ and renormalizing the remaining probabilities:
\begin{equation}
r_k \,=\, \frac{p_k}{1 - p_i} \qquad (k \ne i).
\end{equation}

Consider a single generation slot. We use a shared uniform random variable $U \sim \mathrm{Uniform}(0,1)$ and a categorical draw over non-$i$ items $Z \sim \mathrm{Cat}(\boldsymbol r)$:
\begin{equation}
X^{(p)} \,=\, \begin{cases}
 i, & U \le p, \\
 Z, & U > p.
\end{cases}
\end{equation}
Therefore
\begin{equation}
\begin{aligned}
\mathbb P^{(p)}\{X = i\} &= p, \\
\mathbb P^{(p)}\{X = k\} &= (1 - p)\, r_k \quad (k \ne i).
\end{aligned}
\end{equation}
In particular, for two values $p \in \{p'_i, p''_i\}$:
\begin{equation}
\begin{aligned}
\mathbb P^{(p''_i)}\{X = k\} &= \gamma\, \mathbb P^{(p'_i)}\{X = k\}, \\
\gamma &= \frac{1 - p''_i}{1 - p'_i}.
\end{aligned}
\end{equation}



\section{Note on Batch Differences in Coupled Runs}
\label{app:note-on-batch-difference}

In the coupled runs described in Appendix~\ref{app:popularity-dependence}, the batches differ in ``exclusive slots'' --- slots $b$ where $p'_i < U_b \leq p''_i$.

\subsection{Structure of Differences}

By the coupling construction:
\begin{itemize}
  \item \textbf{Shared slots} ($U_b \leq p'_i$): item $i$ in both runs, identical;
  \item \textbf{Exclusive slots} ($p'_i < U_b \leq p''_i$): item $i$ in run$''$, some item $w \neq i$ in run$'$;
  \item \textbf{Non-$i$ slots} ($U_b > p''_i$): same non-$i$ item in both runs.
\end{itemize}

The number of exclusive slots per batch is $\mathrm{Binomial}(B, p''_i - p'_i)$ with expectation $B \cdot (p''_i - p'_i)$.

\subsection{Impact on Gradient Factors}

When both runs update item $i$ in the same batch, the gradient-magnitude factors $c'$ and $c''$ in \eqref{eq:coupling-increment} may differ due to:
\begin{enumerate}
  \item \textbf{Angular divergence:} The embeddings $q'_i$ and $q''_i$ point in slightly different directions after accumulating different update histories.
  \item \textbf{Batch composition:} In-batch negatives differ in exclusive slots.
\end{enumerate}

Both effects are bounded:
\begin{itemize}
  \item Angular effect: $O(\alpha)$ where $\alpha$ is the angle between $\hat{q}'$ and $\hat{q}''$;
  \item Batch effect: $O(|E_t| / B)$ where $|E_t|$ is the number of exclusive slots.
\end{itemize}

\subsection{Why we explicitly assume a coupled $c$ in the proof}

Because the batch differs in exclusive slots, the scalar factors $c'^{(t)}$ and $c''^{(t)}$ can differ even when both runs update item $i$ in batch $t$.
This matters: if $c''^{(t)} \ll c'^{(t)}$ on a given batch, a \emph{single-step} ordering reversal for the squared norms is possible.

For publication-safe guarantees, Appendix~\ref{app:popularity-dependence} therefore states the monotonicity theorem under an explicit sufficient condition (Assumption ``Coupled $c$'') that isolates the popularity effect as ``how often item $i$ is updated''.

\section{Note on variability of $c$}
\label{app:note-on-c-variability}

Recall the notation from Appendix~\ref{app:cosine-growth-rate}. Let
\begin{equation}
\begin{gathered}
s \,=\, \|q\|^{2}, \qquad P \,=\, I - \hat q\,\hat q^{\!\top}, \\
u \,=\, \sum_{i=1}^{m} \frac{\partial F}{\partial c_i}\,\hat k_i .
\end{gathered}
\end{equation}
With a learning-rate factor $\eta$, a single-update norm increment satisfies
\begin{equation}
\begin{aligned}
\Delta(s) &= \eta^{2}\,\bigl\|\nabla_q L(q)\bigr\|^{2}
          = \frac{\eta^{2}}{s}\,\|Pu\|^{2}
          = \frac{c}{s}, \\
c &= \eta^{2}\,\|Pu\|^{2}.
\end{aligned}
\end{equation}

\paragraph{Why $\boldsymbol{c}$ changes moderately (or sharply, but rarely).\\\\}

\textbf{1) \boldmath$\|u_{t+1}\| \gg \|u_t\|$\unboldmath{} is a rare event:}

In a convergent training process the embeddings move to their optimal positions and the ``error signals'' $\tfrac{\partial F}{\partial cos}$, if they do not decay, at least stabilize. Therefore the norm of the aggregate gradient
\begin{equation}
u \,=\, \sum_{i=1}^{m} \frac{\partial F}{\partial cos_i}\,\hat k_i
\end{equation}
cannot keep growing sharply for many steps; a persistent sharp growth of the norm of the aggregate gradient $u$ would signal a diverging training process.

A sharp growth of the norm of $u$ would require all contributions $\hat k_i$ to add up co-directionally — a worst-case alignment that is statistically suppressed under random batch formation. Regardless of whether this occurs, the sum of contributions is bounded by the number of terms in the batch and by the magnitude of $\tfrac{\partial F}{\partial cos}$, whose stability follows from the argument above.

\textbf{2) If this rare spike happens and temporarily yields \boldmath$s < \sqrt{c}$\unboldmath{} :}

On the same step we have
\begin{equation}
s_{t+1} \,=\, s_t + \frac{c_t}{s_t} \;\ge\; 2\sqrt{c_t} \;\ge\; \sqrt{c_t} ,
\end{equation}
so we immediately re-enter the region where $\Phi'(s) \ge 0$ again. With an increased $s$, violating $\Phi'(s) \ge 0$ again becomes progressively harder.


\end{document}
