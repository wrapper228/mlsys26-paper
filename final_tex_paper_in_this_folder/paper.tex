L% MLSys 2025 paper skeleton (reset)
\documentclass{article}

% Recommended packages
\usepackage{microtype}
\usepackage{graphicx}
\usepackage{subfigure}
\usepackage{amsmath,amssymb}
\usepackage{booktabs}

% MLSys style
\usepackage{mlsys2025}

% Running title (short)
\mlsystitlerunning{Embedding Norm Growth in Two-Tower Models}

\begin{document}

\twocolumn[
\mlsystitle{Theoretical grounds for popularity bias in two-tower models trained with cosine-based loss}

% Keywords are optional
\mlsyskeywords{Representation learning, Two-tower models, Embedding norms}

\vskip 0.1in
\begin{abstract}
    Two-tower models trained with contrastive loss may exhibit popularity bias, observed as a positive correlation between item frequency and its embedding norm that inflates dot-product scores at ranking time. This phenomenon is due to specific properties of loss and encoder architecture. Using analysis of encoder parameters’ updates, I identify sufficient conditions under which embedding updates are provably orthogonal to the current embedding and hence monotonically increase its norm. My theory yields explicit guarantees for the emergence of popularity bias in practical two-tower setups (InfoNCE/cosine-based loss, asymmetric two-tower architecture) and corrects a common misconception in prior works that orthogonality of the gradient alone implies norm inflation for deep encoders.  Empirical studies support the theory via geometry-first investigation: configurations that satisfy these premises exhibit strictly orthogonal embedding movements and a robust statistically significant frequency–norm coupling, whereas violations of any premise break orthogonality and yield non-systematic, noisy update trajectories. The results provide theoretical grounds for popularity bias in cosine-trained two-tower models and show when it should be expected in production systems.
\end{abstract}
]

% Author and affiliation footnote (auto-handled by style in review mode)
\printAffiliationsAndNotice{\mlsysEqualContribution}

% Introduction
\section{Introduction}
Two-tower architectures are the de facto standard for large-scale retrieval and recommendation \cite{huang2024retrievalsurvey}. Their efficiency stems from decoupling the user and item encoders and scoring by a simple similarity, which enables precomputation, approximate nearest-neighbor search, and low-latency serving at web-scale. Yet the role of embedding magnitudes in training and ranking remains under-characterized: empirical observations vary across systems, and theory has not reached consensus on when and why norms change under different objectives and optimizers.

From the definition of cosine similarity, \(\cos(q,k)=\langle q,k\rangle/(\|q\|\,\|k\|)\), the normalization appears to cancel the effect of magnitudes, so objectives that couple the left and right encoder embeddings only through cosine similarity are expected to be insensitive to embedding norms. A competing line of analysis argues \cite{wang2017normface,draganov2024pitfalls,draganov2025embeddingnorms} that for cosine-based losses the gradient \(\nabla_{q}\,\mathcal{L}\) is orthogonal to \(q\), and therefore any gradient step must increase \(\|q\|\), often extrapolated to deep encoders. Both perspectives ignore how parameter updates propagate through the encoder and do not explicitly state the algorithmic and architectural conditions under which norm growth necessarily occurs.

We replace black-box reasoning with an update-level analysis. We track how parameter updates map to item-embedding displacements and prove that, under explicit and testable premises — the item encoder is linear in its parameters with non-shared per-item parameters; the item side is trained by plain SGD; and the loss uses the item embedding only via cosine similarities — each step moves the embedding orthogonally to its current value and thus monotonically increases its squared norm. Building on this, we formalize a frequency--norm \emph{mechanism (conditional on per-update increments)}: popularity increases how often item updates occur, and orthogonality ensures each such update contributes a nonnegative radial increment to $\|q_i\|^2$ (Appendix~\ref{app:popularity-dependence}).

The analysis also delineates the limits of claims that ``gradient orthogonality implies norm inflation'' for deep encoders \cite{wang2017normface,draganov2024pitfalls,draganov2025embeddingnorms}. Orthogonality of the output gradient is insufficient on its own; orthogonality of the \emph{update} follows only when the encoder's Jacobian preserves it, which holds precisely under the stated premises. Violating any premise — adding nonlinearities or parameter sharing in the item tower or optimizing dot product — breaks the guarantee and leads to non-systematic norm dynamics. Our contributions are:

\begin{itemize}
    \item \textbf{Explanation of sufficient conditions for orthogonality of embedding updates.} We prove that 
    embeddings move orthogonally under the following premises: (A1) the item side is optimized by 
    SGD without regularization and momentum; (A2) the item encoder is linear in its parameters; 
    (A3) per-item parameters are not shared; and (A4) the loss depends on the item only via cosine. 
    Orthogonality breaks if any premise is violated.
    \item \textbf{Explanation of why orthogonality of embedding updates implies popularity bias.} We prove a formal mechanism linking sampling frequency to norm growth: under orthogonal updates, each update increases $\|q_i\|^2$, and a clean coupling model yields monotonicity of the expected squared norm with sampling probability. Crucially, 
    we demonstrate that this geometric artifact translates directly into popularity biased retrieval 
    outcomes.
    \item \textbf{Clarification of prior claims in the literature.} We re-examine claims from studies on similar 
    topic \cite{wang2017normface,draganov2024pitfalls,draganov2025embeddingnorms} and argue that gradient 
    update affects the model parameters, not its output itself.
    \item \textbf{The ``Baseline Paradox'' in architectural design.} We identify a structural paradox where 
    simple baselines, favored for their transparency in early development, are mathematically guaranteed to 
    exhibit bias. This creates a risk of false negatives (discarding viable projects due to optimization 
    artifacts), though complex production setups remain vulnerable too.
\end{itemize}

\paragraph{Scope and implications.} We establish sufficient conditions under which embedding updates are orthogonal and popularity bias emerges. Outside this regime, these effects may or may not occur; we make no claim of necessity. Other sources of popularity bias are examined in prior work \cite{zhang2023dimensional}.
\section{Background}
---

\section{Problem statement}
---

\input{sections/related_work}
\section{One formula that underpins the analysis}
---

\section{Key observation about embedding updates}
---

\section{What does parameter-linear encoder mean}
---

\section{Architectures that are parameter-linear}
---

\section{A non-parameter-linear counterexample}
---

\section{When is \(\Delta q_i\) aligned with \(g_i\)}
---

\section{Cosine gradients are orthogonal to outputs}
---

\section{When orthogonal motion appears}
---

\section{Relation to prior work on orthogonality}
---

\section{Norm growth and popularity}
---

\input{sections/mitigations}
\section{Experiments: Empirical Validation of Theoretical Results}
\label{sec:experiments}

\begin{table*}[!t]
\caption{First subsection of experiments — orthogonality of embedding movement across left (user) encoder architectures and losses. See definitions on ``Cond.~1'' and ``Cond.~2'' in subsection definitions.\\}\label{tab:orthogonality-summary}
\centering
\small
\begin{tabular}{@{}cccccc@{}}
\toprule
No. & Left (user) encoder & Loss & \begin{tabular}{@{}c@{}} Cond.~1 for \\ left encoder? \end{tabular} & Cond.~2? & Orthogonal trajectory? \\
\midrule
1 & Embedding & InfoNCE (cos) & $\checkmark$ & $\checkmark$ & $\checkmark$ \\
2 & Embedding & InfoNCE (\textbf{dot}) & $\checkmark$ & $\times$ & $\times$ \\
3 & BERT-like & InfoNCE (cos) & $\times$ & $\checkmark$ & \begin{tabular}{@{}c@{}} User $\times$ / Item $\checkmark$ \end{tabular} \\
4 & Embedding $\to$ Linear & InfoNCE (cos) & $\times$ & $\checkmark$ & \begin{tabular}{@{}c@{}} User $\times$ / Item $\checkmark$ \end{tabular} \\
5 & Embedding (frozen) $\to$ Linear & InfoNCE (cos) & $\times$ & $\checkmark$ & \begin{tabular}{@{}c@{}} User $\times$ / Item $\checkmark$ \end{tabular} \\
6 & one-hot $\to$ Linear & InfoNCE (cos) & $\checkmark$ & $\checkmark$ & $\checkmark$ \\
7 & Embedding & $\bigl(1 - \cos(q,k)\bigr)^{2}$ & $\checkmark$ & $\checkmark$ & $\checkmark$ \\
\bottomrule
\end{tabular}
\end{table*}

To clarify scope, I do not assess recommendation quality (e.g., MAP@k), given that popularity bias can be either harmful or beneficial depending on the application context \citep{klimashevskaia2024popularitybias}. Such evaluation lies outside the contribution of this work. Instead, I empirically examine the full causal chain predicted by the theory: from the orthogonality of updates and frequency–norm coupling (geometric properties) to the resulting skewed retrieval outcomes.
All datasets and implementation details are available in a public \href{https://docs.google.com/anonymous_link_placeholder}{GitHub repository}.

\subsection{Under Which Factors Orthogonality of Embedding Movement Emerges}

See Table~\ref{tab:orthogonality-summary}. In this subsection we use a small toy synthetic dataset in which personalized dependencies can be observed visually.

\textit{Definitions:}\\
Condition~1: the encoder architecture consists of either (i) a single linear layer applied to one-hot inputs, or (ii) a single embedding layer.\\ 
Condition~2: the gradient of the loss with respect to the encoder output is orthogonal to that output.

\textit{Clarifications.} In all experiments the right tower (item tower) is a simple embedding layer (nn.Embedding); InfoNCE \cite{oord2018cpc} denotes the contrastive softmax loss with in-batch negatives, where ``cos'' (resp., ``dot'') refers to cosine (resp., dot-product) similarity. We use SGD without momentum and without regularization; alternative optimizers did not reveal any movement patterns, although we will separately consider mixed-optimizer settings, including Adam, later (see Experiment~10). In Experiment~4 the architecture coincides with one-hot $\to$ Linear $\to$ Linear, i.e., simply two linear layers. The architecture in Experiment~5 is an example of an encoder that consists of a single Linear layer without the input-orthogonality requirement (no one-hot inputs): the inputs are just some numeric features.

\paragraph{If the encoder is nonlinear in the parameters:}
Recall that two linear layers constitute nonlinearity with respect to the model parameters. We argued that due to this nonlinearity one cannot obtain an analytical explanation of how exactly embeddings move. The result of Experiment~4 is consistent with this reasoning: unlike Experiments~1 and~6, here the embeddings from the left encoder move chaotically (see Figure~\ref{fig:exp4-angles}).

\begin{figure}[!htbp]
\centering
\includegraphics[width=.95\columnwidth]{figures/figure_3_paper.pdf}
\caption{Experiment~4: User-update angles distribution on semicircle; black arrow is user embedding, green arrow is orthogonal direction.}
\label{fig:exp4-angles}
\end{figure}

\paragraph{If, in a parameter-linear encoder, the inputs are not mutually orthogonal:}
Recall that linearity of the encoder is still insufficient to guarantee orthogonal movement of embeddings, and consider the result of Experiment~5: embeddings produced by the left encoder move chaotically (see Figure~\ref{fig:exp5-angles}). This agrees with our statement that, for orthogonality of embedding movement, each unique input to the encoder must have its own parameter row that does not intersect with others. We were able to identify only two (essentially identical) architectures where this holds — a linear layer over one-hot vectors (Experiment~6) or an embedding layer (Experiment~1).

\begin{figure}[!htbp]
\centering
\includegraphics[width=.95\columnwidth]{figures/figure_4_paper.pdf}
\caption{Experiment~5: User-update angles distribution on semicircle; black arrow is user embedding, green arrow is orthogonal direction.}
\label{fig:exp5-angles}
\end{figure}

\paragraph{If a different loss is used:}
The result of Experiment~7 does not contradict our statement that the gradient of any cosine-based loss with respect to the encoder output is orthogonal to that output.

Overall, these experiments confirm that strict orthogonality requires precise conditions: violations lead to non-systematic, chaotic movement. For example, the angle distribution for Experiments~4 and~5 (Figures~\ref{fig:exp4-angles} and~\ref{fig:exp5-angles}) shows no strong concentration around $90°$ (share within $\pm 10°$: $\approx$20\%) and spans the full range $[0°, 180°]$.
\subsection{Does Orthogonality of Embedding Movement Imply Popularity Bias?}

See Table~\ref{tab:orthogonality-popbias}. In this subsection I use MovieLens 32M dataset.

In Experiments~8 and~9 I test whether the orthogonality property leads to the emergence of popularity bias. 

\textit{Result.} If embeddings move orthogonally under the factors described above, they do increase their norms; thus items that “receive movement” more often (more popular items, appearing more frequently in training batches) systematically accumulate larger norms. We observe a substantial and statistically significant correlation between item popularity and embedding norm (Pearson correlation 0.54; the null hypothesis of zero correlation is rejected at $p<0.05$). However, there is an important caveat, discussed next.

\textit{Nuance.} We identify an important factor required for the appearance of popularity bias under orthogonal movement: embeddings must move with sufficiently large orthogonal displacement. Purely tangential movement over very short distances yields almost no increase in the embedding norm (by the Pythagorean theorem) and cannot overcome the random norm at initialization. In Experiment~8 I increased the learning rate of the SGD optimizer so that objects moved over larger distances; with a commonly used smaller learning rate (LR $=0.1$) objects moved extremely slowly and effectively did not leave their previous-radius hyperspheres, and I did not observe a statistically significant popularity–norm correlation (Experiment~9).

This finding is consistent with the analysis from Section~\ref{sec:norm-monotonicity}.


\subsection{Popularity bias in practical two-tower setup}

Here I consider an asymmetric two-tower model that is close to a production training design (Experiment 10, Table~\ref{tab:orthogonality-popbias}, same MovieLens 32M dataset): the left tower is a deep BERT over the user's history; the right tower is, again, a simple embedding layer over the item identifier. 
Using such a simple architecture for item tower is a popular scenario when representing an item without convolutions and content features is an intentional design choice. 
For instance, the YouTube paper adopts such an approach by using a trainable item (class) embedding as the input to a softmax loss; 
the CONTEXTGNN paper likewise employs a simple item-embedding matrix as an item tower, arguing that a complex item-side architecture may not provide substantial gains: \emph{"Limited information gain from applying a GNN on the item side"} and \emph{"Shallow embedding matrices are very effective"}.

I use separate optimizers for the two towers: user BERT is trained by Adam with weight decay, and the item embedding layer is trained by SGD without momentum and without regularization. 
In both cases I use a Cyclic LR schedule. 

Empirical results are consistent with the theory: when the right (item) tower and its training dynamics remain simple -- note that a dynamic learning-rate schedule does not invalidate the orthogonal‑movement result -- item embeddings continue to move orthogonally even in the presence of a complex left‑tower architecture.
Given sufficiently large orthogonal displacement (e.g., LR=10 for item tower), a statistically significant popularity bias emerges, observed as a positive correlation between an item’s embedding norm and its frequency in the training set (Pearson correlation 0.56; the null hypothesis of zero correlation is rejected at $p<0.05$).




\input{sections/discussion}
\input{sections/limitations}
\section{Вывод}
Мы показали, что при обучении двухбашенных моделей с косинусными целями возможно систематическое увеличение норм эмбеддингов популярных объектов из-за структуры градиентов и свойств энкодеров. Это влияет на ранжирование при использовании скалярного произведения и требует осознанных мер (архитектура, нормирование, скорая функция).


% References
\bibliography{references}
\bibliographystyle{mlsys2025}

\appendix
\section{Приложения}
Дополнительные выводы, разбор частных случаев и расширенные таблицы/графики будут размещены здесь.


\end{document}
